% Plantilla QA

% Creado por Victor Sanchez
% vsanchezl94@gmail.com
% Marzo 2017

%Formato del título de las secciones
\usepackage{titlesec}
\titleformat*{\section}{\raggedright\bfseries\normalsize}
\titlespacing*{\section}{12pt}{0em}{0em}
\titleformat*{\subsection}{\bfseries\normalsize}

%% Para letra roman
\renewcommand*\rmdefault{ptm}

\makeatletter

	\newcommand{\prof}[1]{\def\@prof{#1}}
	\newcommand{\prep}[1]{\def\@prep{#1}}
	
	\newcommand{\seccion}[1]{\def\@seccion{#1}}
	%ademas
	\newcommand{\muestra}[1]{\def\@muestra{#1}}
	
    \def\@maketitle{%
  %\newpage
  %\null
  \vskip -1em%
  \begin{center}%
  \vskip -1em
  \let \footnote \thanks
    {\normalsize\bf \@title \par\par}%
    \vskip 1em % linea en blanco
    {\normalsize\it\bfseries
	  \@author \par%
    \vskip 1em % linea en blanco
    Profesor: \@prof , Preparador: \@prep \par}%
    \vskip 1em
    {Laboratorio de Química Analítica, sección \@seccion , muestra Nº \@muestra \par%
    
    Escuela de Química, Facultad de Ingeniería, Universidad de Carabobo \par
    \vskip 1em
    Valencia, \@date}

  \end{center}%
  \par}

\makeatother

% Configuracion del resumen (abstract)
\renewcommand{\abstractname}{RESUMEN}
\renewcommand{\abstracttextfont}{\normalfont\normalsize}
\setlength{\absleftindent}{0pt}
\setlength{\absrightindent}{0pt}
\setlength{\abstitleskip}{-12pt}
%\setlength{\absparsep}{0pt}

%Instrucción para evitar la indentación
\setlength\parindent{0pt}
%espacio entre parrafo
\setlength{\parskip}{12pt}